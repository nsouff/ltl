\documentclass{beamer}
\hypersetup{pdfstartview={Fit}}

\usepackage[french]{babel}
\usepackage[T1]{fontenc}
\usepackage[utf8]{inputenc}
\usepackage{eurosym}

\usepackage{amsmath}
\usepackage{amssymb}
\usepackage{amsthm}
\usepackage{mathrsfs}

% Dessiner les automates
\usepackage{tikz}
\usetikzlibrary{automata, positioning, arrows, shapes,shapes.geometric}

% quelques définitions
\theoremstyle{plain}
\newtheorem{thm}{Théorème}
\newtheorem{cor}[thm]{Corollaire}
\newtheorem{lem}[thm]{Lemme}
\newtheorem{prop}{Proposition}

\theoremstyle{definition}
\newtheorem{defi}{Définition}
\newtheorem{rmq}{Remarque}
\newtheorem{ex}{Exemple}
\newtheorem{qst}{Question}

\usetheme{Montpellier}
%\useoutertheme{infolines}
%\usecolortheme{whale}
\beamertemplatenavigationsymbolsempty

\title {Automates et logique temporelle LTL}
\author{
  Souffan Nathan \and
  Bouarah Romain \and
  \\Supervisé par François Laroussinie
}
\date{\today}

\begin{document}

\begin{frame}[plain]
  \titlepage
\end{frame}


\section{Introduction}
\subsection{Sujet}
\begin{frame}
  \frametitle{Le sujet}

  \begin{itemize}
  \item Découvrir la logique LTL
  \item Reconnaître les modèles d'une formule LTL à l'aide d'un automate
  \item Implémenter la construction de cet automate
  \end{itemize}
\end{frame}


\subsection{Vérification de modèles}
\begin{frame}
  \frametitle{Utilité}

  \begin{defi}[Vérification de modèles]
    La vérification de modèles, ou \textit{model checking}, consiste à vérifier certaines propriétés sur le modèle d'un système.
  \end{defi}
  
  \pause
  
  \begin{ex}
    \begin{itemize}[<+->]
    \item On souhaite vérifier la \textit{sûreté et réactivité} d'un \textit{ascenceur}
    \item Vérifier qu'il n'y a \textit{pas d'interblocage} dans un \textit{programme concurrentiel}
    \end{itemize}
  \end{ex}
\end{frame}


\section{LTL}
\subsection{Définitions}

\begin{frame}
  \frametitle{Les logiques temporelles}

  \pause
  Plusieurs logiques temporelles :
  \begin{itemize}[<+->]
  \item \textit{Computation tree logic} (CTL) ou logique du temps arborescent
  \item \textit{linear temporal logic} (LTL) ou Logique temporelle linéaire
  \item \textit{Signal Temporal Logic}, \textit{Metric Interval Temporal Logic}, ...
  \end{itemize}
\end{frame}

\begin{frame}
  \frametitle{Syntaxe}

  Soient $f_1, f_2$ des formules LTL et $p \in AP$ une proposition atomique.
  Une formule LTL $f$ peut s'écrire comme :
  \begin{itemize}
  \item $p$ : atome
  \item $\top$ : tautologie
  \item $\lnot f_1$ : négation
  \item $f_1 \land f_2$ : conjonction
    \pause
  \item $X f_1$ : suivant
  \item $f_1 U f_2$ : jusqu'à
  \end{itemize}  
\end{frame}


\begin{frame}
  \frametitle{Sémantique}

  On interprète une formule sur une position $i \geq 0$ le long d'une exécution étiquetée $(p, l)$ où $p \in Q^\omega$ et $l : Q \to 2^{AP}$ 

  \vfill  
  \begin{tikzpicture}
    \draw[->] (0,0) -- (10, 0) node[left] {};
    \foreach \x in {0,...,4} {\draw (\x,0.1cm) -- (\x,-0.1cm) node[below] {$\x$};}
    \node[below] at (5, 0) {$\dots$};

    \draw (6, 0.1cm) -- (6, -0.1cm) node[below] {$i$};
  \end{tikzpicture}
  \vfill
  
  \pause
  
  \begin{itemize}
  \item $p,l,i \models v \Leftrightarrow v \in l(p(i))$ où $v \in AP$
  \item $p,l,i \models \top$
  \item $p,l,i \models \lnot \varphi \Leftrightarrow p,l,i \not \models \varphi$
  \item $p,l,i \models \varphi \land \psi \Leftrightarrow [(p,l,i \models \varphi) \land (p,l,i \models \psi)]$
  \end{itemize}
\end{frame}

\begin{frame}
  \frametitle{Sémantique de \textit{suivant}}
  \centering
  
  $p,l,i \models X\varphi \Leftrightarrow p,l,i+1 \models \varphi$

  \vspace{1cm}
  
  \begin{tikzpicture}
    \draw[->] (0,0) -- (10, 0) node[left] {};
    \foreach \x in {0,...,2} {\draw (\x,0.1cm) -- (\x,-0.1cm) node[below] {$\x$};}

    \node[below] at (3, 0) {$\dots$};

    \draw (4, 0.1cm) -- (4, -0.1cm) node[below] {$i$};
    \node[above] at (4, 0.1cm) {$X \varphi$};
    
    \draw (5, 0.1cm) -- (5, -0.1cm) node[below] {$i+1$};
    \node[above] at (5, 0.1cm) {$\varphi$};
    
    \node[below] at (6, 0) {$\dots$};
  \end{tikzpicture}
\end{frame}

\begin{frame}
  \frametitle{Sémantique de \textit{jusqu'à}}
  \centering

  $p,l,i \models \varphi U \psi \Leftrightarrow \left[\exists j \geq i \quad \left(p,l,j \models \psi\right) \land \left(\forall i \leq k < j \quad p,l,k\models \varphi\right)\right]$

  \vspace{1cm}
    \begin{tikzpicture}
    \draw[->] (0,0) -- (10, 0) node[left] {};

    \draw (0,0.1cm) -- (0,-0.1cm) node[below] {$0$};    

    \node[below] at (1, -0.1cm) {$\dots$};
    
    \draw (2, 0.1cm) -- (2, -0.1cm) node[below] {$i$};
    \node[above] at (2, 0.1cm) {$\varphi U \psi, \varphi$};
    
    \draw (3, 0.1cm) -- (3, -0.1cm) node[below] {$i+1$};
    \node[above] at (3, 0.1cm) {$\varphi$};

    \node[below] at (4, -0.1cm) {$\dots$};


    \draw (5, 0.1cm) -- (5, -0.1cm) node[below] {$j-1$};
    \node[above] at (5, 0.1cm) {$\varphi$};
    
    \draw (6, 0.1cm) -- (6, -0.1cm) node[below] {$j$};
    \node[above] at (6, 0.1cm) {$\psi$};
  \end{tikzpicture}
\end{frame}



\subsection{Exemples}
\begin{frame}
  \frametitle{Quelques exemples}

  $a, b \in AP$
  \begin{itemize}
  \item $GFa$: (toujours(futur $a$)) ce qui signifie il y a une infinité de positions où $a$ est vrai.
    \pause
  \item $aU(Gb)$: $a$ est vrai tant que $b$ est faux, dès que $a$ est faux, $b$ est toujours vrai par la suite
  \end{itemize}
\end{frame}


\section{Automate de Büchi}
\subsection{Définitions}
\begin{frame}
  \begin{defi}[Automate de Büchi]
    Un automate de Büchi est un quintuplet $\mathcal{A}=(\Sigma, Q, Q_I, \Delta, \mathscr{F})$ où :
    \begin{itemize}
    \item $\Sigma$ est un alphabet
    \item $Q$ est l'ensemble des états
    \item $Q_I \subseteq Q$ est l'ensemble des états initiaux.
    \item $\Delta \subseteq Q \times \Sigma \times Q$ est l'ensemble des transitions.
    \item $\mathscr{F} \subseteq Q$ est l'ensemble des états finaux.
      Un mot $w$ est accepté s'il existe une exécution acceptante de $\mathcal{A}$ sur $w$.
    \end{itemize}
  \end{defi}
\end{frame}

\begin{frame}
  \frametitle{Un exemple d'automate de Büchi}

  Soit $\mathcal{A}=(\{a,b\}, \{q_0, q_1\}, \{q_0\}, \Delta, \{q_1\})$ un automate de Büchi.
  L'ensemble des transitions $\Delta$ est donné dans la figure.
  \begin{figure}
    \centering
    \begin{tikzpicture}[->, >=stealth, node distance=3cm, initial text=$ $, on grid]
      \node[state, initial] (q0) {$q_0$};
      \node[state, accepting, right of=q0] (q1) {$q_1$};
      \draw (q0) edge[bend left, above] node{a} (q1)
      (q1) edge[bend left, below] node{b} (q0)
      (q1) edge[loop above] node{a} (q1);
    \end{tikzpicture}
    \caption{Représentation graphique de $\mathcal{A}$}
  \end{figure}
\end{frame}


\begin{frame}[<+->]
  \begin{defi}[Exécution]
    Soit $w \in \Sigma^\omega$ un mot infini.
    Une exécution de $\mathcal{A}$ sur $w$ est une suite infinie $\rho = q_0q_1q_2\dots \in Q^\omega$ telle que :
    \[
      \forall i \geq 0 \quad (q_i, w_i, q_{i+1}) \in \Delta
    \]
  \end{defi}
  
  \begin{defi}[Exécution acceptante]
    $\rho \in Q^\omega$ une exécution de $\mathcal{A}$ est dite acceptante si :
    \[
      Etats_{\#\infty}(\rho) \cap \mathscr{F} \neq \varnothing
    \]
    où $Etats_{\#\infty}(\rho)$ est l'ensemble des états apparaissants une infinité de fois dans $\rho$.
  \end{defi}
\end{frame}


\begin{frame}
  \frametitle{Un exemple d'automate de Büchi}

  \begin{figure}
    \centering
    \begin{tikzpicture}[->, >=stealth, node distance=3cm, initial text=$ $, on grid]
      \node[state, initial] (q0) {$q_0$};
      \node[state, accepting, right of=q0] (q1) {$q_1$};
      
      \draw (q0) edge[loop, above] node{a} (q0);
      \draw (q0) edge[bend left, above] node{a} (q1);
      \draw (q1) edge[bend left, below] node{b} (q0);
      \draw (q1) edge[loop above] node{a} (q1);
    \end{tikzpicture}
    \caption{Un automate de Büchi}
  \end{figure}

  Pour le mot $a^\omega$, on a :
  \begin{itemize}
  \item $q_0q_0\dots$ qui est une exécution
    \pause
  \item $q_0q_1\dots$ qui est une exécution acceptante
    \pause
  \item ...
  \end{itemize}
\end{frame}

\section{Formule LTL -> automate de Büchi}
\subsection{Définitions}

\begin{frame}
  \frametitle{Automate de Büchi généralisé}
  
  \begin{defi}[Automate de Büchi généralisé]
    Un automate de Büchi généralisé est un quintuplet $\mathcal{A}=(\Sigma, Q, Q_I, \Delta, \mathscr{F})$ où :
    \begin{itemize}
    \item $\Sigma, Q, Q_I, \Delta$ sont comme précédemment.
    \item $\mathscr{F} \subseteq \mathcal{P}(Q)$ est la condition d'acceptation.
      $\mathscr{F}$ est un ensemble d'ensembles finaux. De même, un mot $w$ est accepté s'il existe une exécution acceptante de $\mathcal{A}$ sur $w$.
    \end{itemize}  
  \end{defi}

  $\forall G \in \mathscr{F}$ on passe infiniment de fois par l'un des états de $G$.
\end{frame}


\begin{frame}[<+->]
  \frametitle{Sous formules}
  
  \begin{defi}
    On note $SubF(\varphi)$ l'ensemble des sous formules de $\varphi$ et leur négation.
  \end{defi}

  \begin{ex}
    Si $\varphi = a U b$ alors $SubF(\varphi) = \{ a, \lnot a, b, \lnot b, a U b, \lnot(a U b)\}$.
  \end{ex}
\end{frame}

\begin{frame}[shrink] 
  \begin{defi}[Sous-ensemble cohérent]
    $q \in 2^{SubF(\varphi)}$ est cohérent si :
    \begin{enumerate}[(i)]
    \item $\bot \not \in q$.
    \item Si $\psi_1 \land \psi_2 \in q$ alors $\psi_1 \in q$ et $\psi_2 \in q$.
    \item Si $\psi_1 \lor \psi_2 \in q$ alors $\psi_1 \in q$ ou $\psi_2 \in q$.
    \item $\psi \in q \iff \lnot \psi \not\in q$.
    \end{enumerate}
  \end{defi}

  \pause
  
  \begin{defi}[Sous-ensemble maximal]
    $q \in 2^{SubF(\varphi)}$ est maximal si pour tout $\psi \in SubF(\varphi)$ on a soit $\psi \in q$ soit $\lnot \psi \in q$.  
  \end{defi}

  \pause
  
  \begin{defi}[Sous-ensemble conforme à la sémantique de LTL]
    \label{ss-ens-ok-ltl}
    $q \in 2^{SubF(\varphi)}$ est conforme à la sémantique de LTL :
    \begin{enumerate}[(i)]
    \item Si $\psi_1 U \psi_2 \in q$ alors on a soit $\psi_1 \in q$ soit $\psi_2 \in q$.
    \item $\forall \psi_1 U \psi_2 \in SubF(\varphi)$ si $\psi_2 \in q$ alors $\psi_1 U \psi_2 \in q$.
    \end{enumerate}
  \end{defi}  
\end{frame}


\begin{frame}
  \frametitle{Un exemple}
  
  Soit $\varphi = a U (Xb)$ alors
  \[
    SubF(\varphi) = \{a, \lnot a, b, \lnot b, Xb, \lnot (Xb), a U (Xb), \lnot (aU(Xb))\}
  \]
  \begin{enumerate}
    \pause
  \item $q_1 = \{\lnot a, b, Xb, a U (Xb)\}$ est un sous-ensemble cohérent, maximal et conforme à la sémantique de LTL.
    \pause
  \item $q_2 = \{\lnot a, b, Xb, \lnot(a U (Xb))\}$ est un sous-ensemble cohérent, maximal mais non conforme à la sémantique de LTL car on a $Xb$ et $\lnot(a U (Xb))$.
  \end{enumerate}
\end{frame}


\begin{frame}
  On pose $\mathcal{A}_\varphi = (2^{AP}, Q, Q_I, \Delta, \mathscr{F})$ où :
  \begin{itemize}
  \item $Q \subseteq 2^{SubF(\varphi)}$ contient tout les sous-ensembles cohérents, maximaux et conformes à la sémantique de LTL.
  \item $Q_I = \{ q \in Q | \varphi \in q \}$
  \item $(q, a, q') \in \Delta$ si :
    \begin{enumerate}[(i)]
    \item $\forall p \in AP \quad p \in q \iff p \in a$ (i.e. $a$ possède toutes les propositions atomiques de $q$)
    \item $\forall X\psi \in SubF(\varphi) \quad X\psi \in q \iff \psi \in q'$
    \item $\forall \psi_1 U \psi_2 \in SubF(\varphi) \quad \psi_1 U \psi_2 \in q \iff \left( \psi_2 \in q \lor (\psi_1 \in q \land \psi_1 U \psi_2 \in q')\right)$    
    \end{enumerate}
  \item $\mathscr{F} = \{F_{\psi_1 U \psi_2} | \psi_1 U \psi_2 \in SubF(\varphi)\}$ où
    \[
      F_{\psi_1 U \psi_2} = \{q \in Q | \psi_1 U \psi_2 \not \in q \lor \psi_2\in q \}
    \]
  \end{itemize}
\end{frame}


\begin{frame}[shrink]
  \frametitle{Exemple pour $\varphi = Xa$}
  
  \begin{figure}
    \centering
    \begin{tikzpicture}[->, >=stealth, node distance=3cm, initial text=, on grid, font=\small]
      \tikzstyle{every node}=[rectangle, align=center]
      
      \node[state, initial, accepting, rectangle] (q0) {$\{a, Xa\}$};
      \node[state, accepting, right = of q0, rectangle] (q1) {$\{a, \lnot Xa\}$};
      \node[state, initial, accepting, below = of q0, rectangle] (q2) {$\{\lnot a, Xa\}$};
      \node[state, accepting, below = of q1, right = of q2, rectangle] (q3) {$\{\lnot a, \lnot Xa\}$};

      \draw (q0) edge[loop above] node{$\{a\}$} (q0)
      (q0) edge[above] node{$\{a\}$} (q1)
      (q1) edge[above, anchor=south east, bend left] node{$\{a\}$} (q2)
      (q1) edge[anchor = west] node{$\{a\}$} (q3)      
      (q2) edge[anchor = east] node{$\{\}$} (q0)
      (q2) edge[above, bend left] node{$\{\}$} (q1)
      (q3) edge[above] node{$\{\}$} (q2)
      (q3) edge[loop below] node{$\{\}$} (q3);
      
    \end{tikzpicture}    
    \caption{l'automate de Büchi généralisé pour $Xa$ sur $\{a\}$}
  \end{figure}
\end{frame}

\subsection{Le principal théorème}
\begin{frame}
  \begin{thm}
    \frametitle{Correction de la construction}
    
    Soit $\varphi$ une formule LTL sur $AP$. On a, $\mathcal{L}(\mathcal{A}_\varphi)=mod(\varphi)$ où $mod(\varphi)$ est l'ensemble des modèles reconnues par $\varphi$.
  \end{thm}
\end{frame}
\subsubsection{Lemme 1}
\begin{frame}
	\begin{lem}
		Soit $\omega \in (2^{AP})^\omega$ et $p=q_0q_1\dots$ une exécution acceptante de $\mathcal{A}_\varphi$ du mot $\omega$ alors: \\
		$\forall i \geq 0,\: \forall \psi \in SubF(\varphi): \: (\psi \in q_i \Leftrightarrow \omega_i \models \varphi)$
	\end{lem}
\end{frame}
\begin{frame}
	\begin{proof}
		Par induction structurelle sur $\psi$
		\pause
		\begin{itemize}
			\item $\psi = v \in AP$ \pause
			\item $\psi = \psi_1 \land \psi_2$ \pause
			\item $\psi = \lnot \psi_1$ \pause
			\item $\psi = X\psi_1$ \pause
			\item $\psi = \psi_1 U \psi_2$
		\end{itemize}
	\end{proof}
\end{frame}	


\begin{frame}
  \frametitle{$\mathcal{L}(\mathcal{A}_\varphi) \supseteq mod(\varphi)$}
  
  \begin{lemma}
    Si $w \in (2^{AP})^\omega$ un mot infini sur l'alphabet $2^{AP}$ tel que $w, 0 \models \varphi$
    alors $w \in \mathcal{L}(\mathcal{A}_\varphi)$.
  \end{lemma}

  \pause
  
  \begin{proof}
    \begin{enumerate}[<+->]
    \item $\forall i \geq 0 \quad q_i = \{ \psi \in SubF(\varphi) | w, i \models \psi\}$ et
      $\rho = q_0q_1q_2\dots$
    \item $w, 0 \models \varphi$ donc $\varphi \in q_0$ donc $q_0$ est bien un état initial.
    \item Il y a bien des transition $(q_i, w_i, q_{i+1})$ dans $\mathcal{A}_\varphi$.
    \item si $\psi_1 U \psi_2 \in q_i$, alors $w, i \models \psi_1 U \psi_2$ (par construction des $q_i$)
    donc $\exists j \geq i$ tel que $w,j \models \psi_2$ et $\forall k, i \leq k < j \quad w, k \models \psi_1$.
    Enfin, on a aussi $\psi_1 U \psi_2 \in q_j$ et il existe un chemin valide jusqu'à $q_j$ ainsi $\rho$ passe infiniment souvent par $F_{\psi_1 U \psi_2}$.
    \end{enumerate}    
  \end{proof}
\end{frame}

\section{Présentation du prototype}
\begin{frame}
  
\end{frame}

\end{document}
