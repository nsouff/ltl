\documentclass[12pt,a4paper]{article}

\usepackage[colorlinks]{hyperref}
\usepackage[french]{babel}
\usepackage[T1]{fontenc}

% Formules mathématiques
\usepackage{amsmath}
\usepackage{amssymb}
\usepackage{amsthm}
\usepackage{mathrsfs}

\usepackage[nameinlink]{cleveref}

% Dessiner les automates
\usepackage{tikz}
\usetikzlibrary{automata, positioning, arrows, shapes}

% Ajouter du code
\usepackage[outputdir={rapport_aux}]{minted}




% quelques définitions
\theoremstyle{plain}
\newtheorem{thm}{Théorème}
\newtheorem{cor}[thm]{Corollaire}
\newtheorem{lem}[thm]{Lemme}
\newtheorem{prop}{Proposition}
\newtheorem{dem}{Démonstration}

\theoremstyle{definition}
\newtheorem{defi}{Définition}
\newtheorem{rmq}{Remarque}
\newtheorem{ex}{Exemple}


% Plus d'options sur les listes de type 'enumerate'
\usepackage{enumerate}


\title {Automates et logique temporelle LTL}
\author{
  Souffan Nathan \and
  Bouarah Romain \and
  \\Supervisé par François Laroussinie
}


\begin{document}
\maketitle
\newpage
\tableofcontents
\newpage

\section{Introduction}
Pour vérifier la validité d'un programme ou d'un système il peut être intéressant de connaître quel propriétés doivent être vrais et à quels moments. La logique traditionnelle ne permet de qualifier ce genre d'évènements et il est nécessaire de s'intéresser à une logique dépendant du temps. Avec celle ci on peut ainsi vérifier la validité d'un programme grâce au \emph{model checking} dont on ne parlera pas ici mais qui serait la suite logique de ce document.

\section{Logique temporelle}

Si pour la logique une propriété ne peut être que vraie ou fausse, celle ci peut être vraie à un certain moment puis fausse par la suite en logique temporelle.
Il existe plusieurs représentations du temps pour la logique temporelle : arborescente, continue, linéaire, ...
Pour une représentation linéaire, on peut prendre le temps comme étant $\mathbb{Z}$ pour considérer le passé, ou encore comme ce sera le cas par la suite $\mathbb{N}$.
Celle ci s'appelle la \underline{LTL} (Logique Temporelle de temps Linéaire).

\subsection{LTL}
On se place donc dans le cas d'une logique temporelle linéaire.
On définit $AP$ l'ensemble des propositions atomiques. Il y a alors une valuation des $AP$ pour chaque $n$ dans $\mathbb{N}$.

\subsubsection{Syntaxe}
On définit l'ensemble des formules de LTL comme l'ensemble engendré inductivement et librement par les règles de constructions suivantes:
\begin{itemize}
	\item[] \textbf{Atomes} Si $p \in AP$ alors $p$ est une formule.
	\item[] \textbf{Tautologie} $\top$ est une formule.
	\item[] \textbf{Négation} Si $\varphi$ est une formule propositionnelle, $\lnot \varphi$ en est aussi une.
	\item[] \textbf{Conjonction} Si $\varphi$ et $\psi$ sont des formules, $(\varphi\land \psi)$ est une formule.
	\item[] \textbf{Suivant (Next)} Si $\varphi$ est une formule, $X\varphi$ est une formule.
	\item[] \textbf{Jusqu'à (Until)} Si $\varphi$ et $\psi$ sont des formules, $(\varphi U \psi)$ est une formule.
\end{itemize}

\subsubsection{Sémantique}
On pose $Q=\{q_1, \dots q_n\}$ un ensemble d'états  Pour les formules LTL, les modèles sont des couples $(p, l)$ où $p \in Q^\omega$ et $l : Q \to 2^{AP}$, $l$ nous indique ainsi quels atomes sont vrais  pour chaque état. \\
Les formules sont interprétés sur une position $i \geq 0$ le long d'une exécution étiquetée $(p, l)$. \\
On note ainsi $p, l, i \models \varphi$ le fait que $\varphi$ est vraie en $i$ le long de $(p, l)$, de plus on définit l'équivalence $\equiv$ en posant: $\varphi \equiv \psi \textrm{ si } \forall p,l,i, \:\:[p,l,i \models \varphi \Leftrightarrow p,l,i \models \psi]$
\begin{itemize}
	\item[] $p, l, i \models v \Leftrightarrow v \in l(p(i))$ où $v \in AP$
	\item[] $p,l,i \models \top$
	\item[] $p, l, i \models \varphi \land \psi \Leftrightarrow [(p,l,i \models \varphi) \textrm{ et } (p,l,i \models \psi)]$
	\item[] $p,l,i \models \lnot \varphi \Leftrightarrow p,l,i \not \models \varphi$
	\item[] $p,l,i \models X\varphi \Leftrightarrow p,l,i+1 \models \varphi$
	\item[] $p,l,i \models \varphi\, U \psi \Leftrightarrow [\exists j \geq i\textrm{ tel que, } p,l,i \models \psi \textrm{ et } \forall i \leq j < k, \:\: p,l,k\models \varphi]$
\end{itemize}
\leavevmode \newline 
On ajoute à cela plusieurs macros définis de la sorte:
\begin{itemize}
	\item[] $\bot = \lnot \top$ (Absurde).
	\item[] $\varphi \lor \psi = \lnot (\lnot \varphi \land \lnot \psi)$ (Disjonction).
	\item[] $(\varphi \to \psi) = (\lnot \psi \lor \varphi)$ (Implication).
	\item[] $F\varphi = \top U \varphi$, (Future).
	\item[] $G\varphi = \lnot F \lnot \varphi$ (Globally).

\end{itemize}

\paragraph{Représentation}
\begin{itemize}

	\item[] $\varphi U \psi$ \\
	\shorthandoff{:}
	\begin{tikzpicture}[scale=1]
		\draw[->] (0,0) -- (10.5,0) node[right] {};
		\foreach \x in {0,...,4} {\draw (\x,0.1cm) -- (\x,-0.1cm) node[below] {$\phantom{-}\strut$};}
		\foreach \x in {0,...,4} {\draw (\x, 0.1cm) -- (\x, -0.1cm) node[above] {$\varphi \strut$};}
		\draw (0,0) -- (0,0) node[below]{$i \phantom{-}\strut$};
		\draw (5,0) -- (5,0) node [above] {$\phantom{-}\dots \strut$};
		\draw (8,0.1cm) -- (8,-0.1cm) node [above] {$\psi \strut$};
		\draw (8.2,0) node [below] {$j \phantom{-} \strut$};
		\begin{scope}
			\clip (-2,2) rectangle (2,2);
		\end{scope}
	\end{tikzpicture}\shorthandon{:}\\
	\item[]
\end{itemize}

\begin{prop}
	\leavevmode
	Soient $Q$ un ensemble d'états, $p \in Q^\omega$, $l: Q \to 2^{AP}$ et $i \in \mathbb{N}$, on a:
	\begin{enumerate}
		\item $p,l,i \not \models \bot$
		\item $p,l,i \models \varphi \lor \psi \Leftrightarrow [p,l,i \models \varphi \textrm{ ou } p,l,i \models \psi]$
		\item $p,l,i \models F\varphi \Leftrightarrow p,l,i \models [\exists j \geq i \textrm{ tel que } p,l,j \models \varphi]$
		\item $p,l,i \models G\varphi \Leftrightarrow p,l,i \models [\forall j \geq i \textrm{, on a, } p,l,j \models \varphi]$
	\end{enumerate}
\end{prop}

\begin{proof}
	\leavevmode
	\begin{enumerate}
		\item Trivial
		\item Preuve similaire à celle pour la logique propositionnelle.
		\item 
		\begin{align*}
			p,l,i \models F\varphi &\Leftrightarrow p,l,i \models \top U \varphi \\
			&\Leftrightarrow p,l,i \models [\exists j \geq i \textrm{ tel que } p,l,j \models \varphi \textrm{ et } \forall i \leq j < k, \:\: p,l,k\models \top] \\
			&\Leftrightarrow p,l,i \models [\exists j \geq i \text{ tel que } p,l,j \models \varphi]
		\end{align*}
		\item
		\begin{align*}
			p,l,i \models G \varphi &\Leftrightarrow p,l,i \models \lnot F \lnot \varphi \\
			&\Leftrightarrow p,l,i \not \models F \lnot \varphi \\
			&\Leftrightarrow p,l,i \not \models [\exists j \geq i \textrm{ tel que } p,l,j \models \lnot \varphi] \\
			&\Leftrightarrow p,l,i \not \models [\exists j \geq i \textrm{ tel que } p,l,j \not \models \varphi] \\
			&\Leftrightarrow p,l,i \models [\forall j \geq i \textrm{ on a } p,l,j \models \varphi]
		\end{align*}
	\end{enumerate}
\end{proof}

\begin{ex}
	\leavevmode \newline
	$a, b \in AP$
	\begin{itemize}
		\item $GFa$: (toujours(futur $a$)) ce qui signifie il y a une infinité de positions où $a$ est vrai.
		\item $aU(Gb)$: $a$ est vrai tant que $b$ est faux, dès que $a$ est faux, $b$ est toujours vrai par la suite
	\end{itemize}
\end{ex}
Les propriétés sur les opérateurs de la logique usuelle reste vraies dans la logique temporelle. On peut ajouter des propriétés sur les opérateurs de la logiques temporelle.
\begin{prop}
	\leavevmode
	\begin{enumerate}
		% Negation:
		\item $\lnot (X \varphi) \equiv X(\lnot \varphi)$
		\item $\lnot (G \varphi) \equiv F(\lnot \varphi)$
		\item $\lnot (F \varphi) \equiv G(\lnot \varphi)$

		%Distributivité
		\item $X (\varphi \lor \psi) \equiv (X \varphi) \lor (X \psi)$
		\item $X (\varphi \land \psi) \equiv (X \varphi) \land (X \psi)$
		\item $X (\varphi U \psi) \equiv (X \varphi) U (X \psi)$
		\item $F (\varphi \lor \psi) \equiv (F \varphi) \lor (F \psi)$
		\item $G (\varphi \land \psi) \equiv (G \varphi) \land (G \psi)$
		\item $\xi U (\varphi \lor \psi) \equiv (\xi U \varphi) \lor (\xi U \psi)$
		\item $(\varphi \land \psi) U \xi \equiv (\varphi U \xi) \land (\psi U \xi)$

		%Autres
		\item $F \varphi \equiv FF\varphi$
		\item $G \varphi \equiv GG\varphi$
		\item $\varphi U \psi \equiv \varphi U (\varphi U \psi)$
		\item $\psi \lor (\varphi \land X(\varphi U \psi)) \equiv \varphi U \psi$
		\item $G \varphi \equiv \varphi \land X(G \varphi)$
		\item $F \varphi \equiv \varphi \lor X(F\varphi)$
	\end{enumerate}
\end{prop}

\begin{proof}
	\leavevmode
	\begin{enumerate}
		\item[2.]
		\begin{align*}
			 p,l,i \models \lnot (G \varphi) &\Leftrightarrow p,l,i \not \models G \varphi \\
			 &\Leftrightarrow \lnot (\forall j \geq i\textrm{ tel que } p,l,j \models \varphi) \\
			 &\Leftrightarrow \exists j \geq i \textrm{ tel que } p,l,j \not \models \varphi \\
			 &\Leftrightarrow \exists j \geq i \textrm{ tel que } p,l,j \models \lnot \varphi \\
			 &\Leftrightarrow p,l,i \models F (\lnot \varphi)
		\end{align*}

		\item[] Toutes les preuves reposent sur ce type de démonstration, on laisse les autres en exercices.
	\end{enumerate}
\end{proof}



\section{Automate de Büchi}
Les automates de Büchi sont un type particulier d'automate sur les mots infinis.
Les automates sur les mots infinis (ou $\omega$-automates) sont des automates finis qui acceptent des mots infinis.

\begin{defi}[Automate de Büchi]
  Un automate de Büchi est un quintuplet $\mathcal{A}=(\Sigma, Q, Q_I, \Delta, \mathscr{F})$ où :
  \begin{itemize}
  \item $\Sigma$ est un ensemble fini appelé alphabet de $\mathcal{A}$.
  \item $Q$ est un ensemble fini. Les éléments de $Q$ sont les états de $\mathcal{A}$.
  \item $Q_I \subseteq Q$ est l'ensemble des états initiaux.
  \item $\Delta \subset Q \times \Sigma \times Q$ est l'ensemble des transitions.
  \item $\mathscr{F} \subseteq Q$ est l'ensemble des états finaux (ou états acceptants).
    Un mot $w$ est accepté s'il existe une exécution acceptante de $\mathcal{A}$ sur $w$.
  \end{itemize}
\end{defi}

\begin{defi}[Exécution]
  Soient $w \in \Sigma^\omega$ un mot infini et $\mathcal{A}=(\Sigma, Q, Q_I, \Delta, \mathscr{F})$ un automate de Büchi.
  Une exécution de $\mathcal{A}$ sur $w$ est une suite infinie $\rho = q_0q_1q_2\dots \in Q^\omega$ telle que :
  \[
    \forall i \geq 0 \quad (q_i, w_i, q_{i+1}) \in \Delta
  \]
\end{defi}

\begin{defi}[Exécution acceptante]
  \label{exec-accept}
  Soient $\mathcal{A}=(\Sigma, Q, Q_I, \Delta, \mathscr{F})$ un automate de Büchi et $\rho \in Q^\omega$ une exécution de $\mathcal{A}$.
  On dit que $\rho$ est une exécution acceptante si :
  \[
    Etats_{\#\infty}(\rho) \cap \mathscr{F} \neq \varnothing
  \]
  où $Etats_{\#\infty}(\rho)$ est l'ensemble des états apparaissants une infinité de fois dans $\rho$.
\end{defi}

\begin{defi}[Langage reconnu]
  Soit $\mathcal{A}$ un automate.
  Le langage reconnu par l'automate, noté $\mathcal{L}(\mathcal{A})$, est l'ensemble des mots $w$ tel qu'il existe une exécution acceptante de $w$ sur $\mathcal{A}$.
\end{defi}

\begin{ex}
  Soit $\mathcal{A}=(\{a,b\}, \{q_0, q_1\}, \{q_0\}, \Delta, \{q_1\})$ un automate de Büchi.
  L'ensemble des transitions $\Delta$ est donné dans la figure.
  \begin{figure}[h]
    \centering
    \begin{tikzpicture}[->, >=stealth, node distance=3cm, initial text=$ $, on grid]
      \node[state, initial] (q0) {$q_0$};
      \node[state, accepting, right of=q0] (q1) {$q_1$};
      \draw (q0) edge[bend left, above] node{a} (q1)
      (q1) edge[bend left, below] node{b} (q0)
      (q1) edge[loop above] node{a} (q1);
    \end{tikzpicture}
    \caption{Automate de Büchi}
  \end{figure}
  
  Cet automate reconnaît le mot $w = aaa\dots$ car pour l'exécution $\rho = q_0q_1^\omega$ (c'est la seule) de $\mathcal{A}$ sur $w$, l'état $q_1$ apparaît une infinité de fois dans $\rho$.
  En fait, le langage reconnu (ou accepté) est $a^\omega | a(a^*ba)^\omega | a(a^*ba)^*a^\omega$.
\end{ex}

Les automates de Büchi généralisés sont une variante des automates de Büchi.
La différence se situe sur la condition d'acceptation.
\begin{defi}[Automate de Büchi généralisé]
  Un automate de Büchi généralisé est un quintuplet $\mathcal{A}=(\Sigma, Q, Q_I, \Delta, \mathscr{F})$ où :
  \begin{itemize}
  \item $\Sigma, Q, Q_I, \Delta$ sont comme précédemment.
  \item $\mathscr{F} \subseteq \mathcal{P}(Q)$ est la condition d'acceptation.
    $\mathscr{F}$ est un ensemble d'ensembles finaux. De même, un mot $w$ est accepté s'il existe une exécution acceptante de $\mathcal{A}$ sur $w$.
  \end{itemize}
  
\end{defi}
Pour un automate de Büchi généralisé, une exécution $\rho$ de $\mathcal{A}$ est acceptante si :
\[
  \forall F \in \mathscr{F} \quad Etats_{\#\infty}(\rho) \cap F \neq \varnothing
\]

\section{Traduction de formule LTL en automate de Büchi}
On souhaite ``traduire'' une formule LTL en un automate de Büchi.
Plus précisément, on veut construire un automate qui reconnait les modèles de $\varphi$.

Pour une formule LTL $\varphi$, on peut voir un modèle $(\rho, l)$ de $\varphi$ (où $\rho \in Q^\omega$ et $l : Q \to 2^{AP}$) comme un mot infini sur l'alphabet $2^{AP}$.
Pour cela, il suffit simplement de considérer $(\rho, l)$ comme le mot $l(\rho(1))l(\rho(2))\dots$


\begin{defi}
  On note $SubF(\varphi)$ l'ensemble des sous formules de $\varphi$ et leur négation.
\end{defi}

\begin{ex}
  Si $\varphi = a U b$ alors $SubF(\varphi) = \{ a, \lnot a, b, \lnot b, a U b, \lnot(a U b)\}$.
\end{ex}

\begin{defi}[Sous-ensemble cohérent]
  $q \in 2^{SubF(\varphi)}$ est cohérent si toutes les conditions suivantes sont vérifiées :
  \begin{enumerate}[(i)]
  \item Si $\psi_1 \land \psi_2 \in q$ alors $\psi_1 \in q$ et $\psi_2 \in q$.
  \item Si $\psi_1 \lor \psi_2 \in q$ alors $\psi_1 \in q$ ou $\psi_2 \in q$.
  \item $\psi \in q \iff \lnot \psi \not\in q$.
  \end{enumerate}
\end{defi}

\begin{rmq}
  En utilisant les lois de Morgan, on en déduit aussi que :
  \begin{enumerate}[(i)]
  \item Si $\lnot (\psi_1 \land \psi_2) \in q$ alors $\psi_1 \not \in q$ ou $\psi_2 \not \in q$.
  \item Si $\lnot (\psi_1 \lor \psi_2) \in q$ alors $\psi_1 \not \in q$ et $\psi_2 \not \in q$.
  \end{enumerate}
\end{rmq}


\begin{defi}[Sous-ensemble maximal]
  $q \in 2^{SubF(\varphi)}$ est maximal si pour tout $\psi \in SubF(\varphi)$ on a soit $\psi \in q$ soit $\lnot \psi \in q$.  
\end{defi}

\begin{defi}[Sous-ensemble conforme à la sémantique de LTL]
  \label{ss-ens-ok-ltl}
  $q \in 2^{SubF(\varphi)}$ est conforme à la sémantique de LTL :
  \begin{enumerate}[(i)]
  \item Si $\psi_1 U \psi_2 \in q$ alors on a soit $\psi_1 \in q$ soit $\psi_2 \in q$.
  \item $\forall \psi_1 U \psi_2 \in SubF(\varphi)$ si $\psi_2 \in q$ alors $\psi_1 U \psi_2 \in q$.
  \end{enumerate}
\end{defi}

\begin{ex}
  Soit $\varphi = a U (Xb)$ alors
  \[
    SubF(\varphi) = \{a, \lnot a, b, \lnot b, Xb, \lnot (Xb), a U (Xb), \lnot (aU(Xb))\}
  \]
  \begin{enumerate}
  \item $q_1 = \{\lnot a, b, Xb, a U (Xb)\}$ est un sous-ensemble cohérent, maximal et conforme à la sémantique de LTL.
  \item $q_2 = \{\lnot a, b, Xb, \lnot(a U (Xb))\}$ est un sous-ensemble cohérent, maximal mais non conforme à la sémantique de LTL
    car on a $Xb$ et $\lnot(a U (Xb))$.
  \end{enumerate}
\end{ex}

\begin{defi}
  Soient $AP$ l'ensemble des propositions atomiques et $\varphi$ une formule LTL sur $AP$.
  L'automate de Büchi généralisé pour $\varphi$ sur $AP$ est donné par $\mathcal{A}_\varphi = (2^{AP}, Q, Q_I, \Delta, \mathscr{F})$ où :
  \begin{itemize}
  \item $Q \subseteq 2^{SubF(\varphi)}$ contient tout les sous-ensembles de $SubF(\varphi)$ qui sont cohérents, maximaux et conformes à la sémantique de LTL.
  \item $Q_I = \{ q \in Q | \varphi \in q \}$. Autrement dit, tous les états contenant exactement notre formule de départ $\varphi$ sont des états initiaux.
  \item $\Delta$ est l'ensemble des transitions $(q, a, q')$ avec $q, q' \in Q$ et $a \in 2^{AP}$ vérifiant :
    \begin{enumerate}[(i)]
    \item $\forall p \in AP \quad p \in q \iff p \in a$ (i.e. $a$ possède toutes les propositions atomiques de $q$)
    \item $\forall X\psi \in SubF(\varphi) \quad X\psi \in q \iff \psi \in q'$
    \item $\forall \psi_1 U \psi_2 \in SubF(\varphi) \quad \psi_1 U \psi_2 \in q \iff \left( \psi_2 \in q \lor (\psi_1 \in q \land \psi_1 U \psi_2 \in q')\right)$    
    \end{enumerate}
  \item $\mathscr{F} = \{F_{\psi_1 U \psi_2} | \psi_1 U \psi_2 \in SubF(\varphi)\}$ où
    \[
      F_{\psi_1 U \psi_2} = \{q \in Q | \psi_1 U \psi_2 \not \in q \lor \psi_2\in q \}
    \]
  \end{itemize}
\end{defi}

L'idée dérrière cette construction est de faire en sorte que l'automate devine quelles sont les sous-formules de $\varphi$ qui sont vraies lorsqu'on lit un mot.
Ce sont les états qui indiquent quelles sont les sous formules vraies.

Si l'automate devine que $\lnot p$, avec $p \in AP$, est vraie cela revient à vérifier que $p \not \in a$.
Pour $X \psi$, l'automate vérifie qu'à l'état suivant $\psi$ est bien vraie.
Enfin, si l'on a $\psi_1 U \psi_2$ vérifier $\psi_2$ qu'au prochain état n'est pas suffisant car $\psi_2$ peut se réaliser bien plus loin.
Dans ces cas là, la solution consiste à conserver $\psi_1 U \psi_2$ pour se rappeler qu'on doit encore vérifier cette formule.
Finalement, c'est la condition d'acceptation qui nous permet de dire si la vérification $\psi_1 U \psi_2$ est repoussée indéfiniment ou non.

\newpage
\begin{ex}
  Avec $\varphi = Xa$ on obtient l'automate suivant :
  \begin{figure}[h]
    \centering
    \begin{tikzpicture}[->, >=stealth, node distance=5cm, initial text=$ $, on grid]
      \tikzstyle{every node}=[rectangle, align=center]
      
      \node[state, initial, accepting] (q0) {$\{a, Xa\}$};
      \node[state, accepting, right = of q0] (q1) {$\{a, \lnot Xa\}$};
      \node[state, initial, accepting, below = of q0] (q2) {$\{\lnot a, Xa\}$};
      \node[state, accepting, below = of q1, right = of q2] (q3) {$\{\lnot a, \lnot Xa\}$};

      \draw (q0) edge[loop above] node{$\{a\}$} (q0)
      (q0) edge[above] node{$\{a\}$} (q1)
      (q1) edge[above, anchor=south east, bend left] node{$\{a\}$} (q2)
      (q1) edge[anchor = west] node{$\{a\}$} (q3)      
      (q2) edge[anchor = east] node{$\{\}$} (q0)
      (q2) edge[above, bend left] node{$\{\}$} (q1)
      (q3) edge[above] node{$\{\}$} (q2)
      (q3) edge[loop below] node{$\{\}$} (q3);
      
    \end{tikzpicture}    
    \caption{l'automate de Büchi généralisé pour $Xa$ sur $\{a\}$}
  \end{figure}
  
  Sur cet exemple, une transition $\{\}$ signifie qu'on a $\lnot a$, autrement dit l'absence d'une variable dans une transition signifie que l'on a sa négation.
\end{ex}

On remarque que l'automate peut atteindre une taille exponentielle dans la taille de la formule $\varphi$.
Si on a $Card(SubF(\varphi)) = n$ alors l'automate peut avoir au plus $2^n$ états.

\begin{thm}
	\label{thm1}
  Soient :
  \begin{itemize}
  \item $AP$ un ensemble de propositions atomiques.
  \item $\varphi$ une formule LTL sur $AP$.
  \item $w \in (2^{AP})^\omega$ un mot infini sur l'alphabet $2^{AP}$ tel que $w, 0 \models \varphi$.
  \item $\mathcal{A}_\varphi$ l'automate de Büchi généralisé pour $\varphi$ sur $AP$.
  \end{itemize}
  Alors $w \in \mathcal{L}(\mathcal{A}_\varphi)$. 
\end{thm}

\begin{proof}
  On veut montrer que $w \in \mathcal{L}(\mathcal{A}_\varphi)$.
  D'après la \cref{exec-accept}, cela revient à montrer qu'il existe une exécution acceptante de $w$ sur $\mathcal{A}_\varphi$.

  On pose $\forall i \geq 0 \quad q_i = \{ \psi \in SubF(\varphi) | w, i \models \psi\}$ et $\rho = q_0q_1q_2\dots$
  Montrons que $\rho$ est une exécution acceptante sur $w$ dans $\mathcal{A}_\varphi$.
  \begin{itemize}
  \item $w, 0 \models \varphi$ donc
    $\varphi \in q_0$ donc
    $q_0$ est bien un état initial.
  \item Il y a bien des transition $(q_i, w_i, q_{i+1})$ dans $\mathcal{A}_\varphi$.
  \item $\rho$ vérifie bien la condition d'acceptation.
    En effet, si $\psi_1 U \psi_2 \in q_i$, alors $w, i \models \psi_1 U \psi_2$ (par construction des $q_i$)
    donc $\exists j \geq i$ tel que $w,j \models \psi_2$ et $\forall k, i \leq k \leq j \quad w, k \models \psi_1$.
    Enfin, d'après la \cref{ss-ens-ok-ltl}, on a aussi $\psi_1 U \psi_2 \in q_j$ et il existe un chemin valide jusqu'à $q_j$ ainsi $\rho$ passe infiniment souvent par $F_{\psi_1 U \psi_2}$.
  \end{itemize}
\end{proof}

\begin{thm}
	\label{thm2}
	Soit $\omega \in (2^{AP})^\omega$ et $p=q_0q_1\dots$ une exécution acceptante de $\mathcal{A}_\varphi$ du mot $\omega$ alors: \\
	$\forall i \geq 0,\: \forall \psi \in SubF(\varphi): \: (\psi \in q_i \Leftrightarrow \omega_i \models \varphi)$
\end{thm}
\begin{proof}
	La preuve s'effectue par induction structurelle sur $\psi$\\
	\begin{itemize}
		\item $\psi \in v \in AP$
		\begin{itemize}
			\item[$\bullet$] Si $v \in q_i$, alors par construction de $\mathcal{A}_\varphi$ on sait que $v \in \omega_i$, donc $\omega, i \models v$
			\item[$\bullet$] Inversement si $\omega, i \models v$ alors $v \in \omega_i$ et donc $v \in q_i$ par construction de l'automate.
		\end{itemize}
		\item $\psi = \psi_1 \land \psi_2$
		\begin{itemize}
			\item[$\bullet$] Supposons $\psi_1 \land \psi_2 \in q_i$, alors par construction de $\mathcal{A}_\varphi$ on a $\psi_1, \psi_2 \in q_i$ et donc par hypothèse d'induction on a $\omega, i \models \psi_1$ et $\omega, i \models \psi_2$. Ainsi par définition de la LTL on a $\omega, i \models \psi_1 \land \psi_2$.
			\item[$\bullet$] Inversement, supposons $\omega, i \models \psi_1 \land \psi_2$, alors $\omega, i \models \psi_1$ et $\omega \models \psi_2$ donc par hypothèse d'induction, $\psi_1, \psi_2 \in q_i$. On en conclus par construction de $\mathcal{A}_\varphi$ que $\psi_1 \land \psi_2 \in q_i$
 		\end{itemize}
 		\item $\psi = \lnot \psi_1$
 		\begin{itemize}
 			\item[$\bullet$] Si $\lnot \psi_1 \in q_i$ alors par construction de l'automate $\psi_1 \not \in q_i$. Par hypothèse d'induction on obtient $\omega, i \not \models \psi_1$. Ainsi $\omega, i \models \lnot \psi_1$
 			\item[$\bullet$] Inversement, si $\omega, i \models \lnot \psi_1$. Alors $\omega, i \not \models \psi_1$. Par hypothèse d'induction on obtient donc $\psi_1 \not \in q_i$
 		\end{itemize}
		\item $\psi = X\psi_1$
		\begin{itemize}
			\item[$\bullet$] Supposons $\psi \in q_i$, alors on a $\psi_1 \in q_{i+1}$ par construction de $\mathcal{A}_\varphi$, par l'hypothèse d'induction on a donc $\omega, i+1 \models \psi_1$. Ainsi on a bien par définition $\omega, i \models X\psi_1$
			\item[$\bullet$] Inversement, si $\omega, i \models X\psi_1$ alors $\omega, i+1 \models \psi_1$ et donc par hypothèse d'induction $\psi_1 \in q_{i+1}$. Ainsi par construction de l'automate on a bien $X\psi_1 \in q_i$. 
		\end{itemize}
		\item $\psi = \psi_1 U \psi_2$
		\begin{itemize}
			\item[$\bullet$] Supposons $\psi_1 U \psi_2 \in q_i$, il y a alors deux possibilités.\\
				$\psi_2 \in q_i$: on a alors tout de suite le résultat voulu car $\omega, i \models \psi_2$ et donc $\omega, i \models \psi_1 U \psi_2$. \\
				$\psi_1 \in q_i$ et $\psi_1 U \psi_2 \in q_{i+1}$. Puisque $p$ est une exécution acceptante, il existe une certain $q_j$ avec $j \geq i+1$ et $\psi_2 \in q_j$ (Sinon on aurait $\psi_1 U \psi_2$ dans tous les états $q_j$ où $j \geq i$ mais alors l'exécution ne serait pas bonne), de plus on a $\psi_1 \in q_i,q_{i+1},\dots,q_{j-1}$. Ainsi on en déduit que $\forall i \leq k \leq j , \: \omega, k \models \psi_1 \textrm{ et } \omega, j \models \psi_2$. Par définition de la LTL on en conclus que $\omega, i \models \psi_1 U \psi_2$
			\item[$\bullet$] On suppose maintenant $\omega, i \models \psi_1 U \psi_2$. Alors par définition on a:\\
			$\exists j \geq i$ tel que $\omega, j \models \psi_2$ et $\forall i \leq k < j, \: \omega, k \models \psi_1$\\
			On a alors $\psi_2 \in q_j$ et $\psi_1 \in q_k \:\: \forall i \leq k < j$. \\
			Ainsi on en conclus que $\psi_1 U \psi_2 \in q_i, q_{i+1} \dots, q_j$.
			\item[] On suppose maintenant $\omega, i \models \psi_1 U \psi_2$. Alors par définition on a:\\
			$\exists j \geq i$ tel que $\omega, j \models \psi_2$ et $\forall i \leq k < j, \: \omega, k \models \psi_1$\\
			On a alors $\psi_2 \in q_j$ et $\psi_1 \in q_k \:\: \forall i \leq k < j$. \\
			Ainsi on en conclus que $\psi_1 U \psi_2 \in q_i, q_{i+1} \dots, q_j$.  
		\end{itemize}
	\end{itemize}
	Ainsi le théorème est démontré par induction structurelle.
\end{proof}
\begin{cor}
	Soit $\varphi$ une formule LTL sur $AP$. On a, $\mathcal{L}(\mathcal{A}_\varphi)=mod(\varphi)$ où $mod(\varphi)$ est l'ensemble des modèles reconnues par $\varphi$.
\end{cor}
\begin{proof}
	Par double inclusion en utilisant \cref{thm1} et \cref{thm2}
\end{proof}

\section{Implémentation}
\subsection{Premières définitions des types}
\inputminted[firstline=3, lastline=10]{ocaml}{prototype/ltl.ml}
\inputminted[firstline=3, lastline=3]{ocaml}{prototype/buchi.ml}
\inputminted[firstline=33, lastline=42]{ocaml}{prototype/buchi.ml}

\subsection{Création des états}
\inputminted[firstline=120, lastline=136]{ocaml}{prototype/buchi.ml}
Les sous parties détaillent une à une comment chaque cas est géré.
\subsubsection{Ajout des constante}
\inputminted[firstline=44, lastline=48]{ocaml}{prototype/buchi.ml}
\subsubsection{Ajout des variables}
\inputminted[firstline=50, lastline=59]{ocaml}{prototype/buchi.ml}
\subsubsection{Ajout des OU}
\inputminted[firstline=61, lastline=74]{ocaml}{prototype/buchi.ml}
\subsubsection{Ajout des ET}
\inputminted[firstline=76, lastline=89]{ocaml}{prototype/buchi.ml}
\subsubsection{Ajout des Next}
\inputminted[firstline=91, lastline=99]{ocaml}{prototype/buchi.ml}
\subsubsection{Ajout des Until}
\inputminted[firstline=101, lastline=116]{ocaml}{prototype/buchi.ml}

\subsubsection{Récupération des états initiaux}
\inputminted[firstline=136, lastline=146]{ocaml}{prototype/buchi.ml}
\subsubsection{Récupération des états finaux}
\inputminted[firstline=150, lastline=164]{ocaml}{prototype/buchi.ml}
\subsection{Création des transitions}
Cette fonction vérifie si l'on peut créer une transition de from\_state à to\_state.
\inputminted[firstline=168, lastline=178, breaklines, breakanywhere]{ocaml}{prototype/buchi.ml}
On va essayer de créer toutes les transitions possibles, d'où les deux List.iter pour effectuer un produit cartésien.
\inputminted[firstline=188, lastline=200, breaklines, breakanywhere]{ocaml}{prototype/buchi.ml}



\end{document}
