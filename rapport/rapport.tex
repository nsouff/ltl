\documentclass[12pt,a4paper]{article}

\usepackage[french]{babel}
\usepackage[T1]{fontenc}
\usepackage{amsmath}
\usepackage{amssymb}
\usepackage{amsthm}
\usepackage{tikz}



% quelques définitions
\theoremstyle{plain}
\newtheorem{thm}{Théorème}
\newtheorem{cor}[thm]{Corollaire}
\newtheorem{lem}[thm]{Lemme}
\newtheorem{prop}{Proposition}
\newtheorem{dem}{Démonstration}

\theoremstyle{definition}
\newtheorem{defi}{Définition}
\newtheorem{rmq}{Remarque}
\newtheorem{ex}{Exemple}


\title {Automates et logique temporelle LTL}
\author{
  Souffan Nathan \and
  Bouarah Romain \and
  \\Supervisé par François Laroussinie
}


\begin{document}
\maketitle
\newpage

\section{Logique temporelle}

Si pour la logique une propriété ne peut être que vrai ou fausse, celle ci peut être vrai à un certain moment puis fausse par la suite en logique temporelle. On peut considérer plusieurs espace temps. On peut prendre le temps comme étant $\mathbb{R}$, $\mathbb{Z}$ ou encore comme ce sera le cas par la suite $\mathbb{N}$. Celle ci s'appelle la logique \underline{LTL} (Logique Temporelle Linéaire).

\subsection{LTL}
On se place donc dans le cas d'une logique temporelle linéaire, On défini $AP$ l'ensemble des propositions atomiques. Il y a alors une valuation des $AP$ pour chaque $n$ dans $\mathbb{N}$.

\subsubsection{Syntaxe}
On définit l'ensemble des formules propositionnelles comme l'ensemble engendré inductivement et librement par les règles de constructions suivantes:
\begin{itemize}
	\item[] \textbf{atomes} Si $p \in AP$ alors $p$ est une formule propositionnelle.
	\item[] \textbf{absurde} $\bot$ est une formule propositionnelle.
	\item[] \textbf{Négation} Si $\varphi$ est une formule propositionnelle, $\lnot \varphi$ en est aussi une.
	\item[] \textbf{Conjonction} Si $\varphi$ et $\psi$ sont des formules propositionnelles, $(\varphi\land \psi)$ est une formule propositionnelle.
	\item[] \textbf{Disjonction} Si $\varphi$ et $\psi$ sont des formules propositionnelles, $(\varphi\lor \psi)$ est une formule propositionnelle.
	\item[] \textbf{Implication} Si $\varphi$ et $\psi$ sont des formules propositionnelles, $(\varphi \to \psi)$ est une formule propositionnelle.
	\item[] \textbf{Suivant (next)} Si $\varphi$ est une formule propositionnelle, $X\varphi$ est une formule propositionnelle.
	\item[] \textbf{Jusqu'à (Until)} Si $\varphi$ et $\psi$ sont des formules propositionnelles, $(\varphi U \psi)$ est une formule propositionnelle.
	\item[] \textbf{Toujours (Globally)}. Si $\varphi$ est une formule propositionnelle, $G\varphi$ est une formule propositionnelle.
\end{itemize}

\subsubsection{Sémantique}
On pose $Q=\{q_1, \dots q_n\}$ un ensemble d'états  Pour les formules LTL, les modèles sont des couples $(p, l)$ où $p \in Q^\omega$ et $l : Q \to 2^{AP}$, $l$ nous indique ainsi quels atomes sont vraies  pour chaque états. \\
Les formules sont interprétés sur une position $i \geq 0$ le long d'une exécution étiqueté $(p, l)$. \\
On note ainsi $p, l, i \models \varphi$ le fait que $\varphi$ est vraie en $i$ le long de $(p, l)$, de plus on définit l'équivalence $\equiv$ en posant: $\varphi \equiv \psi \textrm{ si } [p,l,i \models \varphi \Leftrightarrow p,l,i \models \psi]$
\begin{itemize}
	\item[] $p, l, i \models v \Leftrightarrow v \in l(p(i))$ où $v \in AP$
	\item[] $p,l,i \not \models \bot$
	\item[] $p, l, i \models \varphi \land \psi \Leftrightarrow [(p,l,i \models \varphi) \textrm{ et } (p,l,i \models \psi)]$
	\item[] $p,l,i \models \varphi \lor \psi \Leftrightarrow [(p,l,i \models \varphi)\textrm{ ou } (p,l,i \models \psi)]$
	\item[] $p,l,i \models \lnot \varphi \Leftrightarrow p,l,i \not \models \varphi$
	\item[] $p,l,i \models X\varphi \Leftrightarrow p,l,i+1 \models \varphi$
	\item[] $p,l,i \models \varphi\, U \psi \Leftrightarrow [\exists j \geq i\textrm{ tel que, } p,l,i \models \psi \textrm{ et } \forall i \leq j \leq k, \:\: p,l,k\models \varphi]$
	\item[] $\top = \lnot \bot$
	\item[] $F\varphi = \top U \varphi$
	\item[] $G\varphi = \lnot F \lnot \varphi$
	\item[] $(\varphi \to \psi) = (\lnot \psi \lor \varphi)$
\end{itemize}

\paragraph{Représentations}
\begin{itemize}
	
	\item[] $\varphi U \psi$ \\
	\shorthandoff{:}
	\begin{tikzpicture}[scale=1]
		\draw[->] (0,0) -- (10.5,0) node[right] {};
		\foreach \x in {0,...,4} {\draw (\x,0.1cm) -- (\x,-0.1cm) node[below] {$\phantom{-}\strut$};}
		\foreach \x in {0,...,4} {\draw (\x, 0.1cm) -- (\x, -0.1cm) node[above] {$\varphi \strut$};}
		\draw (0,0) -- (0,0) node[below]{$i \phantom{-}\strut$}; 
		\draw (5,0) -- (5,0) node [above] {$\phantom{-}\dots \strut$};
		\draw (8,0.1cm) -- (8,-0.1cm) node [above] {$\psi \strut$};
		\draw (8.2,0) node [below] {$j \phantom{-} \strut$};
		\begin{scope}
			\clip (-2,2) rectangle (2,2);
		\end{scope}
	\end{tikzpicture}\shorthandon{:}\\
	\item[] 
\end{itemize}

\begin{prop}
	\leavevmode 
	\begin{enumerate}
		\item $p,l,i \models \top$
		\item $p,l,i \models F\varphi \Leftrightarrow p,l,i \models [\exists j \geq i \textrm{ tel que } p,l,j \models \varphi]$
		\item $p,l,i \models G\varphi \Leftrightarrow p,l,i \models [\forall j \geq i \textrm{, on a, } p,l,j \models \varphi]$
	\end{enumerate}
\end{prop}

\begin{dem}
	\leavevmode
	\begin{enumerate}
		\item Trivial
		\item 
		\begin{align*}
			p,l,i \models F\varphi &\Leftrightarrow p,l,i \models \top U \varphi \\
			&\Leftrightarrow p,l,i \models [\exists j \geq i \textrm{ tel que } p,l,j \models \varphi \textrm{ et } \forall i \leq j \leq k, \:\: p,l,k\models \top] \\
			&\Leftrightarrow p,l,i \models [\exists j \geq i \text{ tel que } p,l,j \models \varphi]
		\end{align*}
		\item 
		\begin{align*}
			p,l,i \models G \varphi &\Leftrightarrow p,l,i \models \lnot F \lnot \varphi \\
			&\Leftrightarrow p,l,i \not \models F \lnot \varphi \\
			&\Leftrightarrow p,l,i \not \models [\exists j \geq i \textrm{ tel que } p,l,j \models \lnot \varphi] \\
			&\Leftrightarrow p,l,i \not \models [\exists j \geq i \textrm{ tel que } p,l,j \not \models \varphi] \\
			&\Leftrightarrow p,l,i \models [\forall j \geq i \textrm{ on a } p,l,j \models \varphi]
		\end{align*}
	\end{enumerate}
\end{dem}

\begin{ex} 
	\leavevmode \newline
	$a, b \in AP$
	\begin{itemize}
		\item $GFa$: (toujours(futur $a$)) ce qui signifie il y a une infinité de positions où $a$ est vrai.
		\item $aU(Gb)$: $a$ est vrai tant que $b$ est faux, dès que $a$ est faux, $b$ est toujours vrai par la suite
		\item $(a\lor b ) U a \equiv G(a \lor b)$ car la première formule signifie on a $a$ dès qu'on à pas $a\lor b$ donc on doit toujours avoir $a \lor b$
	\end{itemize}
\end{ex}
Les propriétés sur les opérateurs de la logique usuel reste vrais dans la logique temporelle. On peut ajouter des propriétés sur les opérateurs de la logiques temporelle.
\begin{prop} 
	\leavevmode 
	\begin{enumerate}
		% Negation:
		\item $\lnot (X \varphi) \equiv X(\lnot \varphi)$
		\item $\lnot (G \varphi) \equiv F(\lnot \varphi)$
		\item $\lnot (F \varphi) \equiv G(\lnot \varphi)$
		
		%Distributivité
		\item $X (\varphi \lor \psi) \equiv (X \varphi) \lor (X \psi)$
		\item $X (\varphi \land \psi) \equiv (X \varphi) \land (X \psi)$
		\item $X (\varphi U \psi) \equiv (X \varphi) U (X \psi)$
		\item $F (\varphi \lor \psi) \equiv (F \varphi) \lor (F \psi)$
		\item $G (\varphi \land \psi) \equiv (G \varphi) \land (G \psi)$
		\item $\xi U (\varphi \lor \psi) \equiv (\xi U \varphi) \lor (\xi U \psi)$
		\item $(\varphi \land \psi) U \xi \equiv (\varphi U \xi) \land (\psi U \xi)$
		
		%Autres
		\item $F \varphi \equiv FF\varphi$
		\item $G \varphi \equiv GG\varphi$
		\item $\varphi U \psi \equiv \varphi U (\varphi U \psi)$
		\item $\psi \lor (\varphi \land X(\varphi U \psi)) \equiv \varphi U \psi$
		\item $G \varphi \equiv \varphi \land X(G \varphi)$
		\item $F \varphi \equiv \varphi \lor X(F\varphi)$
	\end{enumerate}
\end{prop}

\begin{dem}
	\leavevmode 
	\begin{enumerate}
		\item[2.] 
		\begin{align*}
			 p,l,i \models \lnot (G \varphi) &\Leftrightarrow p,l,i \not \models G \varphi \\
			 &\Leftrightarrow \lnot (\forall j \geq i\textrm{ tel que } p,l,j \models \varphi) \\
			 &\Leftrightarrow \exists j \geq i \textrm{ tel que } p,l,j \not \models \varphi \\
			 &\Leftrightarrow \exists j \geq i \textrm{ tel que } p,l,j \models \lnot \varphi \\
			 &\Leftrightarrow p,l,i \models F (\lnot \varphi)
		\end{align*}	
		
		\item[] Toutes les preuves reposent sur ce type de démonstration, on laisse les autres en exercices.
	\end{enumerate}
\end{dem}







\end{document}